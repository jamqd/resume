%-------------------------------------------------------------------------------
%	SECTION TITLE
%-------------------------------------------------------------------------------
\cvsection{Work Experience (Engineering)}


%-------------------------------------------------------------------------------
%	CONTENT
%-------------------------------------------------------------------------------
\begin{cventries}

  \cventry
    {Undergraduate Student Researcher}
    {Center for Vision, Cognition, Learning, and Autonomy (VCLA) / \newline
    International Center for AI and Robot Autonomy Inc. (CARA)} % Organization
    {Los Angeles, CA} % Location
    {May. 2019 - Present} % Date(s)
    {
    \begin{cvitems} % Description(s) of tasks/responsibilities
        \item {Developing VRGym, an AI research platform for training and evaluating agents in
        3D environments \textbf{(Unreal Engine/ROS/C++/Python)} for causal transfer learning and 
        reinforcement learning research projects. VRGym platform will be open-sourced soon.} 
        \item {Designed and implemented automatic structured, stochastic 3D scene generation including integration with Shapenet and Partnet 
        3D model datasets. 
        Wrote scripts for automated import and conversion of raw 3D model files into Unreal Engine assets.}
        \item {Integrating \textbf{Pyro}, a \textbf{Pytorch}-based probabilistic programming language for probabilistic inference in stochastic VRGym environments.}
        \item {Demonstration of VRGym work was presented by Professor Song-Chun Zhu during an invited talk at \textbf{World AI Conference 2019} in Shanghai.}
    \end{cvitems}
    }

  \cventry
    {Deep Learning Engineer} % Job title
    {Sike AI (Kleiner Perkins Backed Startup)} % Organization
    {Los Angeles, CA} % Location
    {Oct. 2018 - Oct. 2019} % Date(s)
    {
      \begin{cvitems} % Description(s) of tasks/responsibilities
        \item {Created deep learning model for five-factor OCEAN personality trait extraction from text for enabling 
        client companies' to better understand employees. Model predicted 0.0-1.0 valued personality traits \textbf{within 0.015}.
        Utilized state of the art natural language processing algorithms.}
        \item {Designed and implemented data infrastructure, including storage on \textbf{AWS} Simple Storage Service
         and Relational Database Service \textbf{(MySQL)}.
        \item Ran multi-GPU distributed \textbf{Tensorflow} model training on AWS Elastic Compute Cloud, and model deployment on AWS Elastic Beanstalk.}
      \end{cvitems}
    }

  \cventry
  {Undergraduate Student Researcher}
  {Ozcan Research Group (ORG) / Howard Hughes Medical Institute (HHMI)} % Organization
  {Los Angeles, CA} % Location
  {Oct. 2018 - Jun. 2019} % Date(s)
  {
    \begin{cvitems} % Description(s) of tasks/responsibilities
      \item {Trained, evaluated, and tuned deep learning models with \textbf{Tensorflow}. Ran 
      experiments to evaluate various research ideas and visualize results.}
      \item Utilized Tensorflow Lite, model weight quantization and various other efficiency optimizations. Optimizations 
      allowed model to run on Raspberry Pi embedded device, meeting hardware constraints and interfacing with 
      previous work in lab on building lens-free mobile microscopes.
    \end{cvitems}
  }

  \cventry
    {Software Engineering Intern} % Job title
    {Logos News LLC.} % Organization
    {Los Angeles, CA} % Location
    {Oct. 2018 - Dec. 2018} % Date(s)
    {
      \begin{cvitems} % Description(s) of tasks/responsibilities
        \item {Developed iOS app in for diverse, crowd-sourced, and personalized news platform \textbf{(Swift)}. Performed various app bug fixes and refactoring.
        \item Implemented article text highlighting feature enabling text-specific social interaction, discussion, and bias ratings.}
        \item Redesigned Firebase database structure and wrote new Google Cloud Functions, lowering data processing and app loading times \textbf{(Javascript)}.
      \end{cvitems}
    }
\end{cventries}
